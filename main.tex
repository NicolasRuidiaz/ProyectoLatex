\documentclass{article}
\usepackage[utf8]{inputenc}
\usepackage[spanish]{babel}
\usepackage{listings}
\usepackage{graphicx}
\graphicspath{ {images/} }
\usepackage{cite}

\begin{document}

\begin{titlepage}
    \begin{center}
        \vspace*{1cm}
            
        \Huge
        \textbf{Documentación de un proceso específico}
            
        \vspace{0.5cm}
        \LARGE
        Instrucciones para obtener el resultado deseado
            
        \vspace{1.5cm}
            
        \textbf{Nombres y Apellidos del autor}
            
        \vfill
            
        \vspace{0.8cm}
            
        \Large
        Despartamento de Ingeniería Electrónica y Telecomunicaciones\\
        Universidad de Antioquia\\
        Medellín\\
        Marzo de 2021
            
    \end{center}
\end{titlepage}

\tableofcontents
\newpage
\section{Introducción}\label{intro}
El objetivo de estas instrucciones son para que a partir de unos objetos ubicados en cierta posición, usted utilizando solamente una mano logre llevarlos a una posición final de mejor armonía visual, logrando el resultado deseado.

\section{Explicación de la actividad a realizar}\label{explicación}
La posición inicial serán dos tarjetas bajo el centro de una hoja de papel, acostadas una encima de la otra. Usted debe sacar de la parte de abajo de la hoja las tarjetas, y ubicarlas encima y en el centro de la hoja de manera que se sostengan una con la otra formando un triángulo.

\section{Recomendaciones} \label{recomendaciones}
Para poder lograr el resultado final sin mayores inconvenientes usted debe tener en control los siguientes aspectos:
 1.Corrientes de aire: Si se encuentra al interior de una casa, apague el ventilador para que no perturbe la colocación de las tarjetas en la posisión final. Si se encuentra fuera de casa o tiene las ventanas abiertas, busque un lugar con poca brisa.
 2.Movimientos no deseados: Si está en una mesa donde hay mas personas, asegúrese que no muevan el sitio donde tratará de ubicar la hoja y las tarjetas.

\section{Pasos a seguir}\label{pasos}
Levante la hoja y tome las tarjetas. Ubique la hoja en la mesa y las tarjetas acostadas de forma vertical encima de la hoja, para posteriormente con el dedo indice y pulgar(preferiblemente) sostener las tarjetas juntas por el centro de su parte parte superior. A continuación ubique su mano junto con las tarjetas en el centro de la hoja y apoye la tarjeta que quedó sostenida con su dedo índice en la hoja sin soltar el agarre de la otra tarjeta. Luego sin perder el apoyo de la tarjeta con el centro de la hoja, deslice hacia arriba levemente el dedo pulgar para que la segunda tarjeta se separe de la primera y apoyela en el papel buscando el equilibrio de las dos tarjetas para que puedan sostenerse una con la otra (Corrija la postura de las tarjetas levemente con los dedos sin dejarlas caer hasta que logre quitar sus dedos y las tarjetas se sostengan solas) en el centro de la hoja de papel formando un triángulo.
\end{document}